\chapter{Wstęp}
\label{cha:wstep}

\section{Cel pracy}
\label{sec:celpracy}

Celem pracy jest stworzenie demonstratora systemu wizyjnego do śledzenia obiektów, przy czym zakłada się, że kamera zamontowana jest na głowicy obrotowej. Praca inżynierska obejmuje skompletowanie, we współpracy z opiekunem, stanowiska testowego składającego się z głowicy ruchomej, kamery, platformy obliczeniowej oraz wskaźnika. Należy przeprowadzić analizę oraz weryfikację różnych koncepcji śledzenia, biorąc pod uwagę skuteczność oraz możliwość implementacji sprzętowej lub sprzętowo-programowej. Wybrane rozwiązanie zostało zaimplementowane, uruchomione i przetestowane w sprzęcie. Wyjście z modułu śledzenia stanowi podstawę do wypracowania pozycjonowania głowicy obrotowej, takiego, aby utrzymać obiekt w środku kadru oraz oznaczyć go wskaźnikiem.

\section{Wprowadzenie}
\label{sec:wprowadzenie}

Automatyczne śledzenie z pomocą kamery jest wykorzystywane m.in. w zagadnieniach związanych z bezpieczeństwem, inwigilacją lub w zastosowaniach wojskowych. Jest ściśle powiązane z wykrywaniem oraz identyfikacją obiektów. Świadczy to o złożoności tego zagadnienia. Śledzenie różni się od innych algorytmów przetwarzania obrazów głównie tym, że musimy wykorzystywać informacje pochodzące z więcej niż jednego kadru. Uwydatnia się to szczególnie, kiedy kamera rejestruje kilka obiektów podobnych do śledzonego. Jeśli zadaniem jest obserwacja jednego konkretnego obiektu, to aby go wyróżnić musimy wykorzystać położenie śledzonego obiektu w poprzednich ramkach \cite{VT}. Znaczące problemy w śledzeniu obiektów mogą być spowodowane zmianą orientacji celu oraz dużą jego szybkością w porównaniu do częstotliwości rejestrowania klatek. W przypadku kamery umieszczonej na głowicy błędy mogą być dodatkowo spowodowane rozmyciem obrazu w trakcie poruszania układu. Istnieją liczne algorytmy służące śledzeniu elementu na obrazach z kamery. Są to m.in:
\begin{itemize}
\item{Śledzenie przez detekcję}
\item{Mean-shift}
\item{Filtr cząsteczkowy}
\item{KLT}
\end{itemize}
Algorytmy te zostaną dokładniej omówione w kolejnym rozdziale pracy.

\section{Zawartość pracy}
\label{sec:zawartoscpracy}

W rozdziale \dots