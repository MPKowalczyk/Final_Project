\chapter{Podsumowanie}
\label{cha:podsumowanie}

W pracy przeprowadzono analizę algorytmów śledzenia możliwych do zaimplementowania w układzie Zynq. Na podstawie wyników działania implementacji w środowisku \textit{MATLAB} oceniono ich efektywność dla nagranych sekwencji testowych. Najbardziej pod uwagę brana była sekwencja w której śledzonym obiektem był dron. Jako część pracy zostały wybrane części do realizacji stanowiska demonstratora systemu wizyjnego z kamerą zamontowaną na głowicy obrotowej oraz zostało ono zbudowane. Następnie zrealizowano komunikację pomiędzy wszystkimi elementami systemu. Możemy więc zadawać pozycję dla serwomechanizmów bezpośrednio z poziomu komputera PC lub włączyć/wyłączyć pracę autonomiczną. Na podstawie obliczonego modelu matematycznego wybrano nastawy regulatora PID przyrostowego, który służy do pozycjonowania serwomechanizmów. Wartość zadana regulatora jest otrzymywana z toru wizyjnego zaimplementowanego w części rekonfigurowalnej układu. Stworzono dwa tory wizyjne. Jeden z nich realizuje zadanie śledzenia przez detekcję, a drugi algorytm Mean-shift. Każdy z nich na wyjściu daje współrzędne obiektu oraz sygnał wywołujący przerwanie w procesorze. Działanie obu zostało przetestowane w systemie (razemz regulatorem i działającymi serwomechanizmami). Przeprowadzono również próbę implementacji algorytmu KLT.

\paragraph*{}
Wnioski na podstawie wykonanych zadań:
\begin{itemize}
\item Zrealizowanie algorytmu w sposób równoległy (implementacja sprzętowa) wymaga dużo więcej pracy niż implementacja w procesorze.
\item Algorytm Mean-shift radzi sobie dobrze, gdy śledzony obiekt wyraźnie różni się kolorem od tła (jeśli stosujemy składową H przestrzeni HSV).
\item Algorytm śledzenia przez detekcję podawał zazwyczaj dobre współrzędne obiektu, lecz nie wykorzystuje on informacji z poprzednich ramek. Możliwe są więc sytuacje, kiedy wykryje obiekt po zupełnie innej stronie ramki (np. w wyniku szumów), co może doprowadzić nawet do uszkodzeń mechanicznych.
\item Dobór odpowiedniego regulatora i jego parametrów jest dużo łatwiejszy, jeśli dysponujemy modelem sterowanego układu.
\item Model nie musi idealnie odwzorowywać zachowania rzeczywistego układu. Można pominąć efekty takie jak niewielkie opóźnienia.
\item Podczas pracy ze sprzętem trzeba wiele czasu poświęcić na zapoznanie się ze środowiskiem oraz rozwiązywanie powstałych problemów.
\item Podzielenie projektu na niezależne moduły (Rysunek \ref{fig:schemat}) znacznie ułatwia pracę.
\item Testowanie każdego z modułów osobno, po realizacji, ułatwia połączenie układu w działającą całość.
\end{itemize}

Zadania do zrealizowania oraz możliwe ulepszenia systemu:
\begin{itemize}
\item Dokończenie implementacji algorytmu KLT.
\item Przyjęcie modelu śledzonego obiektu.
\item Realizacja filtru Kalmana do filtracji pozycji śledzonego obiektu i przewidywania pozycji, kiedy obiekt jest niewidoczny.
\item Dodanie czujnika pozycji/prędkości serwomechanizmów w celu lepszego pozycjonowania (żyroskop).
\item Zwiększenie częstotliwości klatek z kamery (a więc częstotliwości odczytywania uchybu).
\item Implementacja komunikacji bezprzewodowej z komputerem PC.
\item Analiza innych możliwych do zaimplementowania regulatorów (poprawa sterowania serwomechanizmami).
\end{itemize}