\chapter{Implementacja algorytmu KLT}
\label{cha:implementacjaalgorytmuklt}

%TODO proszę jednak starać się unikać tego "my".
%MK: Żeby je usunąć sprawdiłem wszystkie wystąpienia "my" w pracy.

Jak opisywano w podrozdziale \ref{sec:wyborimplementowanegoalgorytmu}, najlepiej z zadaniem śledzenia w modelu programowym poradził sobie algorytm KLT.
W związku z tym postanowiono przeprowadzić próbę jego implementacji w układzie Zynq. %TODO Stosuje Pan Zynq i ZYNQ - trzeba to ujednolicić. -zrobione.
Należy zauważyć, że celem autora było zaimplementowanie jak największej części algorytmu w logice rekonfigurowalnej, aby zmniejszyć liczbę obliczeń potrzebnych do wykonania w procesorze oraz jak najwięcej z nich wykonywać równolegle. 
W efekcie system ma działać szybko, wymagać małej ilości energii oraz będzie go można łatwo rozbudować lub zmienić zamieniając tylko odpowiednie moduły w diagramie blokowym programu \textit{Vivado}.

\paragraph*{}
Metoda śledzenia KLT działa na obrazie w skali szarości.
Konwersję przeprowadzono tak, jak dla metody Mean-shift, z użyciem wzoru (\ref{eq:grayscale}). 
Zgodnie z przebiegiem algorytmu przestawionym w podrozdziale \ref{sec:klt}, działanie algorytmu rozpoczyna się od wyboru punktów charakterystycznych spełniających warunek na wartości własne hesjanu. 
%TODO sekcja - w PL to nie jest rodział, b) H, czy hesjan -zrobione.
%MK: Zmieniłem sekcje na podrozdziały. hesjan
Aby to zrobić najpierw należy wyliczyć gradient obrazu wejściowego. Jest on obliczany jako konwolucja dwuwymiarowa obrazu wejściowego z następującymi maskami:
\begin{equation}
M_x=
	\begin{bmatrix}
	-1 & 0 & 1
	\end{bmatrix}
\end{equation}
\begin{equation}
M_y=
	\begin{bmatrix}
	-1 \\
	0 \\
	1
	\end{bmatrix}
\end{equation}

Wyniki powyższych splotów są następnie podnoszone do kwadratów i mnożone przez siebie. 
W wyniku otrzymuje się macierz hesjanu dla aktualnego piksela. Zdecydowano, że powyższe wartości będą wyliczane dla całego obrazu w sposób potokowy.
Aby to zrealizować należy zaimplementować również linię opóźniającą wartości pikseli oraz sygnały synchronizacyjne z użyciem wewnętrznej pamięci blokowej (BRAM), gdyż do wykonania powyższych działań potrzebne są wartości pikseli z wcześniejszych linii obrazu wejściowego.
Należy również wyznaczyć latencję użytych sumatorów i mnożarek, by opóźnić sygnały synchronizacyjne o odpowiednią liczbę taktów zegara.

\paragraph*{}

Aby wyznaczyć punkty charakterystyczne następnie należy wyliczone wartości hesjanów poddać filtracji filtrem Gaussowskim. 
Jądro dwuwymiarowego filtru Gaussowskiego o wymiarze \(3\)x\(3\), wartości oczekiwanej równej \((0,0)\) i odchyleniu standardowym równym \(1\) wynosi:
\begin{equation}
\label{eq:gauss}
M_G=
	\begin{bmatrix}
	0.0751 & 0.1238 & 0.0751\\
	0.1238 & 0.2042 & 0.1238\\
	0.0751 & 0.1238 & 0.0751
	\end{bmatrix}
\end{equation}

Zwykły filtr Gaussowski wymagałby jednak dużej liczby zasobów (kilku mnożarek), więc postanowiono zaimplementować tzw. dyskretną wersję powyższego jądra, która ma postać \cite{Benda}:
\begin{equation}
M_G=\frac{1}{16} \cdot
	\begin{bmatrix}
	1 & 2 & 1\\
	2 & 4 & 2\\
	1 & 2 & 1
	\end{bmatrix}
\end{equation}
Powyższy filtr w porównaniu do (\ref{eq:gauss}) nie wymaga użycia mnożarek. Zamiast tego wykorzystuje przesunięcia bitowe. Implementacja wymaga kolejnej linii opóźniającej oraz dodatkowego opóźnienia sygnałów synchronizacyjnych.

\paragraph*{}
Na podstawie wyników opisanego powyżej splotu oblicza się funkcję celu, czyli wartość mniejszej wartości własnej hesjanu. Hesjan ma postać:
\begin{equation}
H=
	\begin{bmatrix}
	H_{xx} & H_{xy}\\
	H_{xy} & H_{yy}
	\end{bmatrix}
\end{equation}
Wyznaczono analitycznie jego wartości własne.
\begin{equation}
\begin{vmatrix}
H_{xx}-\lambda & H_{xy} \\
X_{xy} & H_{yy}-\lambda
\end{vmatrix}=0
\end{equation}
\begin{equation}
(H_{xx}-\lambda)(H_{yy}-\lambda)-H_{xy}^2=0
\end{equation}
\begin{equation}
\lambda^2-\lambda(H_{xx}+H_{yy})+H_{xx}H_{yy}-H_{xy}^2=0
\end{equation}
\begin{equation}
\label{eq:delta}
\Delta=(H_{xx}-H_{yy})^2+4H_{xy}^2 \geqslant 0
\end{equation}
Mniejsza z jego wartości własnych wynosi więc:
\begin{equation}
\label{eq:lambda}
\lambda=\frac{H_{xx}+H_{yy}-\sqrt{\Delta}}{2}
\end{equation}
Piksele dla których obliczona wartość funkcji celu jest największa są punktami charakterystycznymi. Na podstawie wzorów (\ref{eq:delta}) i (\ref{eq:lambda}) wnioskować można, że obliczenie wartości własnej wymagać będzie modułów mnożarki, sumatorów oraz obliczania pierwiastka podanej wartości.
%TODO styl - ze wzorów "widać" -zrobione.
Obliczenia te zostały zrealizowane za pomocą modułów dostępnych w programie \textit{Vivado} (tzw. IP Core'ów). Na rysunku \ref{fig:mineig_vivado} przedstawiono fragment schematu blokowego, który odpowiada za wyznaczanie punktów charakterystycznych.

\begin{figure}[H]
	\centering
	\includegraphics[width=4in]{mineig_vivado.jpg}
	\caption{Fragment schematu blokowego, realizujący wyznaczanie punktów charakterystycznych.}
	\label{fig:mineig_vivado}
\end{figure}
%TODO ref. i nie, że powyżej opisane operacje... -zrobione.

\begin{figure}[H]
	\centering
	\includegraphics[width=2in]{test.jpg}
	\caption{Przykładowy obraz wejściowy.}
	\label{fig:test}
\end{figure}
%TODO ref -zrobione.

\begin{figure}[H]
	\centering
	\includegraphics[width=4in]{test_out.jpg}
	\caption{Przykładowy obraz wyjściowy.}
	\label{fig:test_out}
\end{figure}
%TODO ref -zrobione.

\noindent Na rysunku \ref{fig:test} pokazano przykładowy obraz wejściowy dla którego wyznaczone zostały punkty charakterystyczne. Został on wybrany tak, by ich położenie można było bardzo łatwe do określenia (znajdują się one w rogach prostokątów).\newline
Na rysunku \ref{fig:test_out} przedstawiono obraz wyjściowy, gdy na wejście został podany obraz z rysunku \ref{fig:test}. Jasne piksele oznaczają punkty charakterystyczne. Zgodnie z przewidywaniami znajdują się one w rogach prostokątów.

\paragraph*{}
Nie udało się dokończyć implementacji algorytmu KLT. Potrzeba jeszcze zaimplementować operacje realizujące śledzenie okna wokół wybranych punktów charakterystycznych, a więc obliczające wartość przesunięcia z równania (\ref{eq:dp_klt}) oraz iteracyjne wyznaczanie przesunięcia za pomocą wspomnianego wzoru.

%TODO Zobaczymy...