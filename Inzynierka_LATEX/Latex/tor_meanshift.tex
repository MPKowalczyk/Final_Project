\chapter{Tor wizyjny z algorytmem Mean-shift}
\label{cha:torwizyjnyzalgorytmemmeanshift}

Przekształcono tor wizyjny zbudowany w rozdziale \ref{cha:torwizyjny} tak, by źródłem położenia obiektu był gotowy algorytm Mean-shift, który został opisane w sekcji \ref{sec:meanshift}. Rozpoczęto od przystosowania otrzymanego modułu tak, by bardziej nadawał się do celów projektu. Najpierw zmieniono sposób podawania rozmiaru okna oraz początkowego jego położenia z inicjalizacji rejestrów odpowiednimi wartościami na parametry modułu. Dzięki temu podczas testów nie trzeba generować ponownie modułu w celu zmiany powyższych parametrów. Oprócz tego dodano sygnał wyjściowy informujący o wyznaczeniu nowego położenia środka obiektu. Używany on jest do wysyłania współrzędnych środka do procesora oraz do zgłoszenia przerwania w procesorze.

\paragraph*{}
Użyty algorytm Mean-shift wylicza wartości prawdopodobieństwa dla pikseli na podstawie histogramu ich otoczenia i histogramu początkowej ramki. Działa on na obrazie w skali szarości. Jego działanie może być więc w dużej mierze zależne od użytej metody konwersji przestrzeni barw RGB. Postanowiono najpierw użyć wartości składowej Y z przestrzeni barw YUV. Obliczana jest ona następująco:
\begin{equation}
\label{eq:grayscale}
Y=0.299 \cdot R+0.587 \cdot G+0.114 \cdot B
\end{equation}
Bierze się w ten sposób pod uwagę, że oko ludzki jest najbardziej wyczulone na kolor zielony, a najmniej na kolor niebieski. Do wykonania powyższej konwersji zostały użyte 3 moduły mnożarki w układzie FPGA.
Oprócz wartości piksela i sygnałów synchronizacyjnych na wejście podawany jest również sygnał ustawienia histogramu obiektu. Zapisywany jest wtedy histogram obiektu aktualnie znajdującego się w oknie. Sygnał ten został podłączony do przełącznika znajdującego się na płytce ZYBO. Wyjście modułu realizującego śledzenie zostało wyprowadzone do modułu, który na obrazie wyjściowym zaznacza aktualne okno.

\paragraph*{}
Najpierw sprawdzono działanie zaimplementowanego systemu podając na wejście HDMI obraz z komputera PC. Było nim czarne koło na białym tle. Przesuwano koło i sprawdzano, czy zaznaczone okno będzie się przesuwać razem z nim. Na podstawie wyników stwierdzono, że algorytm działa, lecz gubi obiekt, kiedy ten porusza się zbyt szybko, lub wykona jakiś gwałtowny ruch. Może to być bardzo problematyczne w trakcie śledzenia rzeczywistego obiektu. Dla obrazu z kamery nie będzie również tak dobrze rozróżniony obiekt od tła. Można się więc domyślać, że algorytm ten dużo gorzej będzie działać w rzeczywistym układzie.

\paragraph*{}
Następnie na wejście HDMI podano obraz z kamery używanej w projekcie. Śledzono zielony prostokątny element. Problemem okazał się wspomniany brak dobrego rozróżnienia obiektu od tła. Zmiana położenia obiektu, a w efekcie zmiana oświetlenia powodowała zmianę śledzonego obiektu na element otoczenia.

\paragraph*{}
Aby zwiększyć efektywność testowanego algorytmu zmieniono sposób wyznaczania obrazu w skali szarości. Zamiast obliczania składowej Y przestrzeni YUV do modułu śledzącego podano składową H przestrzeni barw HSV. Odpowiada ona za kolor obiektu, więc powinna dać lepsze wyniki gdy obiekt ma inny kolor niż otoczenie \cite{Kryjak}.
\begin{equation}
V=max(R,G,B)
\end{equation}
\begin{equation}
S=
	\begin{cases}
	\frac{V-min(R,G,B)}{V} &\text{jeśli } V \neq 0\\
	0 &\text{jeśli } V=0
	\end{cases}
\end{equation}
\begin{equation}
H=
	\begin{cases}
	0 &\text{jeśli } V-min(R,G,B)=0\\
	\frac{60(G-B)}{V-min(R,G,B)} &\text{jeśli } V=R\\
	\frac{60(B-R)}{V-min(R,G,B)}+120 &\text{jeśli } V=G\\
	\frac{60(R-G)}{V-min(R,G,B)}+240 &\text{jeśli } V=R
	\end{cases}
\end{equation}
\begin{equation}
H \leftarrow
	\begin{cases}
	\frac{H+360}{360} &\text{jeśli } H<0\\
	\frac{H}{360} &\text{jeśli } H \geqslant 0
	\end{cases}
\end{equation}
Zmiana ta podyktowana jest faktem, że śledzony obiekt jest koloru zielonego, a w otoczeniu występuje niewiele elementów tego koloru. W wyniku opisanej zmiany algorytm Mean-shift zaczął dużo lepiej śledzić testowy obiekt. Gubiony był on w wyniku zbyt nagłego ruchu obiektu lub gdy zmieniał się kolor tła.